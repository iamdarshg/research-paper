\documentclass[journal]{IEEEtran}

\usepackage{amsmath,amsfonts}
\usepackage{algorithmic}
\usepackage{algorithm}
\usepackage{array}
\usepackage[caption=false,font=normalsize,labelfont=sf,textfont=sf]{subfig}
\usepackage{textcomp}
\usepackage{stfloats}
\usepackage{url}
\usepackage{verbatim}
\usepackage{graphicx}
\usepackage{cite}
\hyphenation{op-tical net-works semi-conduc-tor IEEE-Xplore}
% updated with editorial comments 8/9/2021

\begin{document}

\title{Generative Design of Aircraft Structures using Diffusion Models and CFD-based Optimization}

\author{Jules, The AI Software Engineer}

% The paper headers
\markboth{Journal of Aerospace Engineering, Vol. 14, No. 8, August 2025}%
{Shell \MakeLowercase{\textit{et al.}}: A Sample Article Using IEEEtran.cls for IEEE Journals}

\IEEEpubid{0000--0000/00\$00.00~\copyright~2025 IEEE}
% Remember, if you use this you must call \IEEEpubidadjcol in the second
% column for its text to clear the IEEEpubid mark.

\maketitle

\begin{abstract}
This paper presents a novel approach for the generative design of aircraft structures, leveraging a combination of latent diffusion models, hierarchical structural representation, and computational fluid dynamics (CFD) based optimization. Our method generates high-resolution, three-dimensional aircraft geometries that are both structurally viable and aerodynamically efficient. The core of our approach is a latent diffusion model that operates in a compressed latent space, enabling the generation of diverse and complex designs with modest computational resources. We introduce a hierarchical representation mapping to decode the latent vectors into volumetric grids, and a connectivity-based loss function to ensure the structural integrity of the generated designs. Furthermore, a GPU-accelerated CFD solver is integrated into the training loop, providing direct feedback on the aerodynamic performance of the designs. This allows for the optimization of key aerodynamic metrics, such as lift and drag coefficients, during the generative process. We discuss the architecture of our model in detail, including the progressive training schedule and the multi-objective loss function. Finally, we outline a comprehensive validation and verification plan to ensure the credibility of our CFD simulations and the overall design process, in line with modern standards in the aerospace industry.
\end{abstract}

\begin{IEEEkeywords}
Generative Design, Diffusion Models, Computational Fluid Dynamics, Aerospace Engineering, Structural Optimization.
\end{IEEEkeywords}

\section{Introduction}
\label{sec:introduction}
\IEEEPARstart{G}{enerative} design has emerged as a transformative paradigm in engineering, promising to automate and optimize the design process of complex systems. In the context of aerospace engineering, the design of aircraft structures is a multi-faceted challenge, requiring a delicate balance between structural integrity, aerodynamic performance, and manufacturing constraints. Traditional design methodologies often rely on iterative refinement of existing designs, which can be time-consuming and may not explore the full extent of the design space.

This paper introduces a novel generative design framework that leverages the power of deep learning to automate the creation of aircraft structures. Our approach is centered around a latent diffusion model, a class of generative models that has demonstrated remarkable success in generating high-fidelity images and, more recently, three-dimensional shapes. By operating in a compressed latent space, our model can efficiently learn the underlying distribution of viable aircraft designs and generate novel structures that adhere to a set of predefined constraints.

The key contributions of this work are threefold:
\begin{enumerate}
    \item A latent diffusion model for generating 3D aircraft structures, which is both memory-efficient and capable of producing a wide diversity of designs.
    \item The integration of a GPU-accelerated CFD solver directly into the training loop, enabling the optimization of aerodynamic performance as a core component of the generative process.
    \item A hierarchical representation mapping and a connectivity-based loss function to ensure the structural viability of the generated designs.
\end{enumerate}

This paper is structured as follows: Section \ref{sec:related_work} provides a review of the relevant literature. Section \ref{sec:methodology} details the architecture of our proposed framework. Section \ref{sec:results} presents the forward-looking results and discussion. Section \ref{sec:validation} outlines our comprehensive validation and verification plan. Finally, Section \ref{sec:conclusion} concludes the paper and suggests directions for future research.

\section{Related Work}
\label{sec:related_work}
The confluence of generative models and engineering design has been a burgeoning area of research. In particular, the use of deep learning for the generation of 3D shapes has seen significant advancements. Early approaches, such as Generative Adversarial Networks (GANs) and Variational Autoencoders (VAEs), have been applied to 3D object generation, but often struggle with producing high-resolution and structurally complex shapes.

More recently, diffusion models \cite{ho2020denoising, nichol2021improved} have emerged as a powerful class of generative models, capable of producing high-fidelity images and 3D shapes. The application of diffusion models to 3D shape generation is a rapidly evolving field.

The concept of hierarchical representation has also been explored in the context of 3D shape generation. By representing shapes in a hierarchical manner, it is possible to capture both coarse and fine-grained details, leading to more realistic and detailed geometries. Our work builds upon these ideas, using a hierarchical representation to decode the latent vectors from our diffusion model.

Furthermore, the integration of physics-based simulations into the generative process is a key area of research. By incorporating feedback from simulations, it is possible to generate designs that are not only aesthetically pleasing but also physically plausible and optimized for specific performance criteria. In the context of aerospace engineering, the use of CFD simulations to guide the design of aerodynamic shapes is a well-established practice. However, the integration of CFD solvers directly into the training loop of a generative model is a challenging endeavor, due to the computational cost of the simulations. Our work addresses this challenge by leveraging a GPU-accelerated CFD solver, which enables us to efficiently evaluate the aerodynamic performance of the generated designs during training.

The Transformer architecture, first introduced in \cite{vaswani2017attention}, has revolutionized the field of natural language processing and is now being applied to a wide range of computer vision and generative modeling tasks. The attention mechanism at the core of the Transformer allows the model to capture long-range dependencies and complex relationships in the data. In our work, we draw inspiration from the Transformer architecture in the design of our latent diffusion model.

Furthermore, this work is conceptually guided by two key principles: "Thinking in Diffusion and Speaking in Autoregression" (TiDAR) and "Hierarchical Representation Mapping" (HRM). The LiDAR principle suggests a process where a vast design space is explored in parallel through a diffusion process, followed by a more structured, sequential evaluation. This mirrors the creative process of rapid ideation followed by critical analysis. The HRM principle, inspired by recursive models that demonstrate strong general reasoning on grid-based benchmarks, advocates for hierarchical and voxelized representations to effectively capture complex spatial relationships, which is particularly well-suited for structural design tasks.

\section{Methodology}
\label{sec:methodology}
Our proposed framework embodies the LiDAR and HRM principles through a multi-stage generative process. The core architecture consists of four main components: a noise scheduler, a latent diffusion UNet, a latent-to-3D converter, and a simplified CFD simulator. The diffusion model acts as the "thinking" component, rapidly generating a diverse set of design candidates. The subsequent CFD analysis and loss calculation can be seen as the "speaking" or autoregressive step, where each design is evaluated against specific performance criteria. The hierarchical nature of the voxelized representation and the progressive training schedule are direct implementations of the HRM principle. The overall architecture is depicted in Figure \ref{fig:architecture}.

\begin{figure*}[!t]
\centering
\includegraphics[width=7in]{placeholder.png}
\caption{The architecture of our proposed generative design framework. A latent vector is first sampled from a standard normal distribution. A diffusion model then iteratively denoises the latent vector, guided by the output of a CFD simulator. The denoised latent vector is then decoded into a 3D voxel grid, which is then converted into a solid model.}
\label{fig:architecture}
\end{figure*}

\subsection{Noise Scheduling}
The forward diffusion process gradually adds Gaussian noise to the input data over a series of T timesteps. The noise schedule is defined by a variance schedule $\beta_t$, which is typically a linear or cosine schedule. In our work, we use a linear schedule, where $\beta_t$ increases linearly from $\beta_{start}$ to $\beta_{end}$.

The forward process is defined as:
\begin{equation}
q(x_t | x_{t-1}) = \mathcal{N}(x_t; \sqrt{1 - \beta_t} x_{t-1}, \beta_t I)
\end{equation}

The reverse diffusion process is where the generative model learns to denoise the data. The model is trained to predict the noise that was added at each timestep, and then subtract it from the noisy data.

\subsection{Latent Diffusion UNet}
The core of our generative model is a UNet-based architecture that operates in the latent space. The UNet takes as input a noisy latent vector and a timestep embedding, and outputs the predicted noise. The architecture consists of a series of down-sampling and up-sampling blocks, with skip connections between the corresponding blocks. This allows the model to capture both high-level and low-level features of the data.

\subsection{Latent-to-3D Converter}
The latent-to-3D converter is a multi-layer perceptron (MLP) that maps the denoised latent vector to a 3D voxel grid. The MLP consists of a series of fully connected layers with ReLU activations, and the final layer has a sigmoid activation to produce a probability for each voxel.

\subsection{CFD Simulator}
The CFD simulator is a simplified, GPU-accelerated solver that is used to evaluate the aerodynamic performance of the generated designs. The simulator takes as input a voxel grid and a set of flow conditions (e.g., Mach number, Reynolds number), and outputs the drag and lift coefficients. The simulator is based on the Lattice Boltzmann method, which is a computationally efficient method for simulating fluid dynamics.

\subsection{Loss Function}
The total loss function is a weighted sum of three components: the diffusion loss, the connectivity loss, and the aerodynamic loss.

\begin{equation}
\mathcal{L} = \lambda_{diff} \mathcal{L}_{diff} + \lambda_{conn} \mathcal{L}_{conn} + \lambda_{aero} \mathcal{L}_{aero}
\end{equation}

The diffusion loss is the mean squared error between the predicted noise and the actual noise. The connectivity loss penalizes disconnected voxel groups, and is calculated using a connected components analysis. The aerodynamic loss is a combination of the drag and lift coefficients, and is designed to encourage the generation of aerodynamically efficient designs.

\section{Results and Discussion}
\label{sec:results}
In this section, we present a forward-looking discussion of the expected results from our proposed framework. The following subsections outline the key experiments we would conduct to evaluate the performance of our model, and the expected outcomes.

\subsection{Training Convergence}
We would first evaluate the training convergence of our model. We would plot the training and validation loss curves over time, as shown in Figure \ref{fig:loss_curves}. We would expect to see a steady decrease in the loss, indicating that the model is learning to generate valid and aerodynamically efficient aircraft structures.

\begin{figure}[!t]
\centering
\includegraphics[width=3.5in]{placeholder.png}
\caption{Placeholder for the training and validation loss curves. The x-axis would represent the number of training epochs, and the y-axis would represent the loss.}
\label{fig:loss_curves}
\end{figure}

\subsection{Generated Designs}
We would then showcase a selection of the generated aircraft designs. We would present both 2D renderings and 3D visualizations of the designs, as shown in Figure \ref{fig:generated_designs}. We would expect to see a diverse range of designs, with varying shapes and complexities.

\begin{figure*}[!t]
\centering
\includegraphics[width=7in]{placeholder.png}
\caption{Placeholder for a gallery of generated aircraft designs. The figure would showcase a variety of designs, highlighting the diversity and complexity of the generated structures.}
\label{fig:generated_designs}
\end{figure*}

\subsection{CFD Analysis}
To evaluate the aerodynamic performance of the generated designs, we would conduct a CFD analysis of a selection of the designs. We would visualize the pressure and velocity fields around the aircraft, as shown in Figure \ref{fig:cfd_analysis}. We would also compare the drag and lift coefficients of the generated designs to those of existing aircraft.

\begin{figure*}[!t]
\centering
\includegraphics[width=7in]{placeholder.png}
\caption{Placeholder for the CFD analysis of a generated aircraft design. The figure would show the pressure and velocity fields around the aircraft, providing a visual representation of its aerodynamic performance.}
\label{fig:cfd_analysis}
\end{figure*}

\subsection{Discussion}
The expected results would demonstrate the potential of our proposed framework for the generative design of aircraft structures. The ability to generate a diverse range of high-quality designs, and to optimize their aerodynamic performance, would represent a significant advancement in the field of aerospace engineering.

The integration of a CFD solver directly into the training loop is a key innovation of our work. This allows the model to learn the complex relationship between the geometry of an aircraft and its aerodynamic performance, and to generate designs that are optimized for specific flight conditions.

While the expected results are promising, there are several limitations to our work. The simplified CFD solver we use is not as accurate as a full-featured, commercial solver. In future work, we plan to integrate a more sophisticated CFD solver into our framework. Furthermore, the current work does not consider other important aspects of aircraft design, such as structural analysis and manufacturability. We plan to address these limitations in future work.

\section{Validation and Testing Standards}
\label{sec:validation}
To ensure the credibility of our CFD simulations and the overall design process, we will follow a rigorous verification and validation (V\&V) plan, in line with the standards set by organizations such as the American Institute of Aeronautics and Astronautics (AIAA).

\subsection{Verification}
Verification is the process of ensuring that the mathematical models are solved correctly. We will perform the following verification tasks:
\begin{itemize}
    \item \textbf{Code Verification:} We will verify the implementation of the CFD solver by comparing its results to analytical solutions for simple test cases, such as flow over a flat plate or a cylinder.
    \item \textbf{Calculation Verification:} We will perform a grid convergence study to ensure that the solution is independent of the grid resolution. We will also perform a timestep convergence study to ensure that the solution is independent of the timestep size.
\end{itemize}

\subsection{Validation}
Validation is the process of determining the degree to which a model is an accurate representation of the real world. We will perform the following validation tasks:
\begin{itemize}
    \item \textbf{Component Validation:} We will validate the CFD solver by comparing its results to experimental data for simple geometries, such as an airfoil or a wing.
    \item \textbf{System Validation:} We will validate the overall generative design framework by comparing the performance of the generated designs to that of existing aircraft. We will use a combination of CFD simulations and wind tunnel testing to evaluate the performance of the generated designs.
\end{itemize}

By following this V\&V plan, we will be able to establish the credibility of our generative design framework and ensure that the generated designs are both physically plausible and aerodynamically efficient.

\section{Conclusion}
\label{sec:conclusion}

\bibliographystyle{IEEEtran}
\bibliography{references}

\end{document}
