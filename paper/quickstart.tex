\documentclass[12pt]{article}
\usepackage[utf8]{inputenc}
\usepackage{graphicx}
\usepackage{amsmath}
\usepackage{hyperref}
\usepackage{geometry}
\geometry{margin=1in}
\usepackage{listings}
\usepackage{xcolor}

\lstset{
    language=Python,
    basicstyle=\ttfamily\small,
    keywordstyle=\color{blue},
    commentstyle=\color{gray},
    stringstyle=\color{red},
    breaklines=true,
    showstringspaces=false
}

\title{Quick Start Guide: Paper Airplane AI Optimization}
\author{Darsh Gupta}
\date{December 2025}

\begin{document}

\maketitle

\section{Installation}

\subsection{System Requirements}
\begin{itemize}
    \item Python 3.8+
    \item NVIDIA GPU (RTX 3060 or better, 6GB+ VRAM)
    \item Windows 10/11 (Linux/macOS supported)
    \item 4GB RAM minimum
\end{itemize}

\subsection{Step 1: Clone Repository}
\begin{lstlisting}
git clone <repository_url>
cd research-paper
\end{lstlisting}

\subsection{Step 2: Install Dependencies}
\begin{lstlisting}
pip install -r requirements.txt
pip install torch-geometric  # For GNN support
\end{lstlisting}

\subsection{Step 3: (Optional) Install FluidX3D}

Download from \url{https://www.fluidx3d.com/} and run the Windows installer. This enables high-fidelity CFD evaluation:

\begin{lstlisting}
# Verify installation
where FluidX3D
\end{lstlisting}

If not installed, the system automatically falls back to physics-based surrogate model.

\section{Quick Start}

\subsection{Launch Web GUI}
\begin{lstlisting}
python -m streamlit run src/gui/app.py
\end{lstlisting}

This opens an interactive Streamlit interface at \texttt{http://localhost:8502}.

\subsection{Graphical Interface Walkthrough}

\subsubsection{Sidebar: Device Configuration}
\begin{itemize}
    \item Select GPU device (auto-detected all NVIDIA GPUs)
    \item View VRAM and CUDA capability
    \item Configure number of folds, target range, training episodes
    \item Load/save trained models
\end{itemize}

\subsubsection{Tab 1-3: Example Designs}
\begin{itemize}
    \item Example 1: Classical dart airplane design
    \item Example 2: AI-optimized design
    \item Example 3: Experimental asymmetric design
    \item Each tab shows performance metrics and distribution graphs
\end{itemize}

\subsubsection{Tab 4: Training \& Validation}
\begin{itemize}
    \item \textbf{CFD Method Selector}: Choose between surrogate (fast), FluidX3D (accurate), or hybrid
    \item \textbf{Training Method}: DDPG (reinforcement learning) or GNN (pattern learning)
    \item \textbf{Hyperparameters}: Episodes, batch size, learning rate
    \item \textbf{Real-time Monitoring}: Loss curves, metrics, 3D mesh visualization
    \item \textbf{Batch Evaluation}: Test 100-1000 designs with distribution analysis
\end{itemize}

\section{Understanding CFD Methods}

\subsection{Surrogate Model (Fast)}
\begin{itemize}
    \item \textbf{Speed}: 0.1 seconds per design
    \item \textbf{Accuracy}: ~75\% (compared to CFD)
    \item \textbf{Method}: Physics-based equations (lifting line theory + corrections)
    \item \textbf{Best For}: Optimization loops, exploring design space
    \item \textbf{Memory}: <100MB
\end{itemize}

\subsection{FluidX3D (High-Fidelity)}
\begin{itemize}
    \item \textbf{Speed}: 10-20 seconds per design (GPU-accelerated)
    \item \textbf{Accuracy}: ~95\% (validated CFD)
    \item \textbf{Method}: Lattice Boltzmann Method (LBM) on GPU
    \item \textbf{Best For}: Validating designs, final performance prediction
    \item \textbf{Memory}: 2-4GB VRAM
    \item \textbf{Requires}: NVIDIA GPU + FluidX3D installation
\end{itemize}

\subsection{Hybrid Approach (Auto-Select)}
\begin{itemize}
    \item \textbf{Speed}: ~5 seconds average per design
    \item \textbf{Accuracy}: ~90\%
    \item \textbf{Strategy}: Use surrogate for exploration, FluidX3D for promising designs
    \item \textbf{Best For}: Balanced optimization (accuracy vs speed)
\end{itemize}

\section{Training Methods}

\subsection{DDPG Agent (Reinforcement Learning)}
\begin{itemize}
    \item \textbf{Algorithm}: Deep Deterministic Policy Gradient
    \item \textbf{Training Time}: 5-10 minutes for 100 episodes
    \item \textbf{Best For}: Direct optimization, single objective (range)
    \item \textbf{Output}: Fold sequence that maximizes range
\end{itemize}

Usage:
\begin{lstlisting}
1. Launch GUI
2. Tab 4: Select "DDPG Agent (RL-based)"
3. Set episodes (50-100 recommended)
4. Click "Start Training"
\end{lstlisting}

\subsection{Recursive GNN (Graph Neural Networks)}
\begin{itemize}
    \item \textbf{Algorithm}: 3-level hierarchical graph attention network
    \item \textbf{Training Time}: 2-5 minutes for 50 epochs
    \item \textbf{Best For}: Pattern recognition, transfer learning
    \item \textbf{Output}: Learned mapping from geometry to aerodynamics
\end{itemize}

Usage:
\begin{lstlisting}
1. Launch GUI
2. Tab 4: Select "Recursive GNN (Pattern Learning)"
3. Set epochs (30-50 recommended)
4. Click "Start Training"
\end{lstlisting}

\section{Workflow Examples}

\subsection{Fast Optimization (2 minutes)}
\begin{lstlisting}
1. Tab 4: Select "Surrogate Model (Fast)" CFD
2. Select "DDPG Agent" training
3. Set 30 episodes
4. Click "Start Training"
5. View results and trained design
\end{lstlisting}

\subsection{High-Fidelity Validation (5-10 minutes)}
\begin{lstlisting}
1. Tab 4: Select "FluidX3D (High-Fidelity)" CFD
2. Select "DDPG Agent" training
3. Set 20 episodes (fewer, more accurate)
4. Click "Start Training"
5. Get validated design performance
\end{lstlisting}

\subsection{Pattern Discovery (5 minutes)}
\begin{lstlisting}
1. Tab 4: Select "Recursive GNN" training method
2. Select CFD method (surrogate recommended)
3. Set 40 epochs
4. Click "Start Training"
5. Analyze learned patterns and predictions
\end{lstlisting}

\subsection{Batch Analysis (2 minutes)}
\begin{lstlisting}
1. Tab 4: Scroll to "Batch Evaluation"
2. Set 500 designs
3. Click "Run Batch Evaluation"
4. View distribution histograms
5. Identify best performers
\end{lstlisting}

\section{Expected Results}

\subsection{Performance Metrics}

\subsubsection{Example Designs}
\begin{itemize}
    \item Classical Dart: L/D $\approx$ 8-10, Range $\approx$ 18m
    \item AI-Optimized: L/D $\approx$ 12-15, Range $\approx$ 22m
    \item Experimental: L/D $\approx$ 15-18, Range $\approx$ 25m (FluidX3D validated)
\end{itemize}

\subsubsection{Training Progress}
\begin{itemize}
    \item Episode 1-10: Random designs, poor performance
    \item Episode 20-50: Improvement visible, convergence begins
    \item Episode 50-100: Diminishing returns, near-optimal range achieved
\end{itemize}

\section{Troubleshooting}

\subsection{GPU Not Detected}
\begin{lstlisting}
# Check NVIDIA drivers
nvidia-smi

# If not found, install NVIDIA CUDA Toolkit
# Verify PyTorch sees GPU:
python -c "import torch; print(torch.cuda.is_available())"
\end{lstlisting}

\subsection{FluidX3D Not Found}
\begin{lstlisting}
# Install from https://www.fluidx3d.com/
# Verify:
where FluidX3D

# System automatically uses surrogate fallback if not available
\end{lstlisting}

\subsection{Out of Memory (OOM)}
\begin{itemize}
    \item Reduce batch size (32 instead of 64)
    \item Reduce number of training episodes
    \item Use CPU instead of GPU (slower but works)
    \item Close other GPU applications (Chrome, games, etc.)
\end{itemize}

\subsection{Training Very Slow}
\begin{itemize}
    \item Verify GPU is being used (Tab 4 shows device name)
    \item Check task manager for GPU utilization
    \item Select smaller episode count for testing
    \item Use surrogate instead of FluidX3D
\end{itemize}

\section{Next Steps}

\subsection{Explore}
\begin{itemize}
    \item Run all three example tabs to see designs
    \item Train DDPG agent with different configurations
    \item Experiment with GNN training
    \item Analyze batch evaluation results
\end{itemize}

\subsection{Develop}
\begin{itemize}
    \item Modify hyperparameters in \texttt{config.yaml}
    \item Edit fold constraints in \texttt{src/folding/}
    \item Customize reward function in \texttt{src/rl\_agent/env.py}
    \item Add new metrics or visualization
\end{itemize}

\subsection{Validate}
\begin{itemize}
    \item Compare surrogate vs FluidX3D predictions
    \item Validate learned patterns against theory
    \item Test designs in wind tunnel (optional)
    \item Publish results
\end{itemize}

\section{Support}

\begin{itemize}
    \item \textbf{Documentation}: See \texttt{ImplementationDetails.tex} for technical details
    \item \textbf{Code}: Comments throughout \texttt{src/} explain key functions
    \item \textbf{Issues}: Check \texttt{README.md} for known issues
    \item \textbf{Contact}: Refer to main paper for citations and contact info
\end{itemize}

\end{document}
